% Prior to using this template, please update the appropriate
% style files used.
\documentclass[pdftex,10pt,conference,letterpaper]{}

\usepackage[T1]{fontenc}
\usepackage{lmodern}                    % Don't use Computer Modern
\usepackage{pslatex}                    % This uses postscript fonts
\usepackage[dvips]{graphicx}            % Useful for inserting images
\usepackage{siunitx}                    % For SI units
\usepackage[cmex10,intlimits]{amsmath}  % For the math
\usepackage{amsthm}                     % The package used for theorems
\usepackage{amssymb}                    % Math symbols
\usepackage{amsfonts}                   % For script R and other symbols
%\usepackage[hidelinks,
%pagebackref=false]{hyperref}            % Make references clickable
%\usepackage[all]{hypcap}                % Figures visible on click
\usepackage[final,tracking=true,
kerning=true,spacing=true,
factor=1100]{microtype}                 % For overall prettiness
\usepackage{booktabs}                   % For pretty tables.
\usepackage{array}                      % For pretty tables.

% On Microtype:
% Intro: http://www.khirevich.com/latex/microtype/
% Manual: http://mirror.math.ku.edu/tex-archive/macros/latex/contrib/microtype/microtype.pdf

\input{./macrosN.tex}           % Some useful macros

\begin{document}
%
% paper title
% can use linebreaks \\ within to get better formatting as desired
\title{}

\author{}

% make the title area
\maketitle

\begin{abstract}
%\boldmath
%    \input{./tex/abstract.tex}
\end{abstract}

% For peerreview papers, this IEEEtran command inserts a page break and
% creates the second title. It will be ignored for other modes.
%\IEEEpeerreviewmaketitle

\section{Introduction}\label{sec:intro}
%\input{./tex/intro.tex}

\section{Related Work}\label{sec:related}
%\input{./tex/relatedW.tex}

\section{Evaluation}\label{sec:evaluation}
%\input{./tex/evaluation.tex}

\section{Conclusion}\label{sec:conclusion}
%\input{./tex/conclusion.tex}

%\bibliographystyle{IEEEtran}
%\bibliography{IEEEabrv,refs}

% that's all folks
\end{document}
